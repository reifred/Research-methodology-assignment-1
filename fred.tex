\documentclass[a4paper,10.9pt]{article}
\begin{document}
\title{LEARNING BASIC SWIMMING SKILLS.}
\author{MUGERWA FRED 15/U/7866/PS 215004276}
\date{\today}
\maketitle
\section{Abstract}
\paragraph{
The aim of this report is to teach people how to learn basic swimming skills through a step by step process. More so it helps people to knowing advantages associated with having swimming skills than having no such skills.}
\section{Introduction}
\paragraph{I truly believe swimmers are not born, but are made. Swimming is a skill that can help keep you fit and can be a good exercise for your muscles. You burn many calories when swimming and it can be useful during emergency situations. It can also help in reducing drowning accidents in the world both between the young and the adults. }
\paragraph{When you start a swimming program, you may find that it is difficult to complete a lap and you're quite breathless, even if you are already fit. This is because swimming requires controlled breathing when your face is in the water, which takes time to learn }
\paragraph{This research was conducted based on experiment and steps I used when learning how to swim. 
To progress as a swimmer, you need to take lessons, but you can start learning to swim by trying a number of things on your own.}
\section{Steps for learning basic swimming skills.}
\paragraph{Get in the water and walk around the shallow end. Use this time to get used to the feel of the water and how buoyant you are. Progress to deeper water, getting wet up to your armpits or shoulders. Many beginners have a natural fear of water. Don't worry if this takes a few visits until you're comfortable going this far into the pool.}
\paragraph{Get in the shallow end. Hold onto the side of the pool. Put your face in the water and blow bubbles. Stand back up and breathe normally. Practice until you feel comfortable with your face in the water. Move to deeper water and, without holding onto the side, put your face in the water and blow bubbles. Stand back up and breathe normally. Practice until you feel comfortable.}
\paragraph{Learn to float. Hold the side of the pool. Take a deep breath and lift your feet up while leaning backward. Try to float. This can take a few tries. Practice until you can float for 15 to 30 seconds. Practice without holding onto the side. }
\paragraph{Hold onto the side. Take a deep breath and put your face in the water while kicking your feet out behind you. Try to float. Practice until you can float for 10 to 15 seconds. Practice without holding onto the side.}
\paragraph{Grab a flotation device and try one lap of swimming. A kickboard or Styrofoam noodle is ideal. Don't use arm supports or anything around your waist as these types of flotation devices interfere rather than help. Hold the kickboard in front you with straight arms. Push off from the wall and scissor kick with your legs straight behind you, rotating your head to the side to breathe. Complete one lap, resting along the way as necessary.}
\paragraph{Try another lap, including your arms. Hold the flotation device in front of you with straight arms. Push off from the wall and scissor kick with straight legs. Lift one arm off the kickboard, pulling down through the water and lifting, returning your arm to its starting position. Switch arms and repeat. Rotate your head to breathe from the side as necessary. Rest and repeat as you feel comfortable.}
\paragraph{Try a lap without the flotation device. Push the kickboard out in front of you and begin your freestyle stroke, scissor kicking, straight arms in front and side breathing. When you reach the kickboard, either use it as support to rest or push it out again and continue for another few strokes. Rest and repeat as you feel comfortable.}
\section{Conclusion}
\paragraph{Swimming is a necessary skill among people due to the advantages associated with it. It is both important for body health and emergency situations such as drowning so people should make an effort to have at least basic swimming skills.}
\end{document}